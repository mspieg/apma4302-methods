\documentclass[12pt]{article}
\usepackage{times}
\usepackage[margin=1in]{geometry}
\usepackage{amsmath}
\usepackage{amssymb}
\usepackage{amsthm}
\usepackage{graphicx}
\usepackage{fancyhdr}

% Header setup
\pagestyle{fancy}
\fancyhf{}
\rhead{Name: \underline{\hspace{3cm}}}
\lhead{APMA 4302 - Methods}
\cfoot{\thepage}

\begin{document}

\title{APMA 4302 Methods - Homework 1}
\author{Marc Spiegelman}
\date{\today}
\maketitle

\section*{Problem 1}

A slightly different modification of problem 1.1 from the Beuler textbook.

Modify the code \texttt{expx.c} to compute a well-balanced $N$ term Taylor Polynomial approximation
for $e^x$ for both positive and negative values of $x$. The command line should read
\begin{verbatim}
    $ mpiexec -n nP ./expx -x x -N N 
\end{verbatim}
where \texttt{x} is the point at which to evaluate $e^x$, \texttt{N} is the number of terms
in the Taylor polynomial, and \texttt{nP} is the number of processes to use.

The code should have the following properties:
\begin{itemize}
\item you should use \texttt{PETSc} options handling to read in the command line arguments.
\item The Taylor polynomial should be computed in parallel using \texttt{nP} processes.
\item For each process, the work should be roughly $O(N/nP)$.
\item The code should work for both positive and negative values of \texttt{x}.
\item The answer should be printed out from process 0 only, and should include the relative error
compared to the value computed by the \texttt{exp()} function from the standard C, in multiples of machine precision. 
\item You can use the constant \texttt{PETSC\_MACHINE\_EPSILON} to get machine precision.
\end{itemize}

\end{document}